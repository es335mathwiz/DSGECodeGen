%"C:\Program Files\Wolfram Research\Mathematica\9.0\math.exe"
%\documentclass[pdf]{beamer}
%set BIBINPUTS=.;../../bibFiles;;
%set TEXINPUTS=.;../../texFiles;../../bibFiles;;
\documentclass{beamer}
\mode<presentation>{}
\usepackage{beamerthemeshadow}
\usepackage[utf8]{inputenc}
\usepackage{listings,xcolor}


\usepackage{moreverb}
\usepackage{ulem}
\usepackage{soul}
%\usepackage{algorithmicx}
%\usepackage{algpseudocode}
\usepackage{pseudocode}
\usepackage{hyperref}
% \usepackage{color}
% \definecolor{light}{gray}{.80}
% \usepackage{graphicx}
% \usepackage{amsfonts}
% \usepackage{amsmath}
% \usepackage{amssymb}
% \usepackage{amsthm}
% \usepackage{algorithmicx}
%\usepackage{program}
% \usepackage{rotating}
\usepackage{ulem}
\usepackage{authordate1-4}
\newcommand{\dsgecg}{{\em DSGECodeGen}}
\newcommand{\pdrv}[2]{\frac{\partial #1}{\partial #2}}
\newcommand{\mma}{Mathematica}

\begin{document}
\title[Symbolic Algebra/XML and Code Generation]{\dsgecg: An Application of Symbolic Algebra and XML to DSGE Code Generation}
%\subtitle{this is a subtitle}


\author{Gary S. Anderson}
\date{\today} 


\frame{\titlepage}


\section{Overview}



\begin{frame}
  \frametitle{Overview}
  
  \begin{itemize}
  \item \dsgecg\  Objectives and Components
  \item XML for Model Definition and Manipulation
  \item Symbolic Algebra for Code Generation
  \item Examples
    \begin{itemize}
    \item Anisotropic Grid Smolyak
    \item Anderson Moore Algorithm 
    \end{itemize}
  \item Future Directions
  \end{itemize}
\end{frame}

\section{\dsgecg\  Objectives and Components}

\begin{frame}
  \frametitle{Objectives and Components}  

{\small
\begin{itemize}
\item Apply Modern software development techniques 
Symbolic Algebra Tools to Generate Portable Interoprable Compilable Code  Robust
Error Free
  \begin{itemize}
  \item Automate Error Prone Tasks
  \item Dynare other Personal Libraries
  \item Interface Design
  \item Open Source Tools  (sympy)  XML
  \item Not replacement for Dynare or even personal Libraries
  \item Target Languages
  \item Interoperability and code sharing
  \item promote specialization
  \item object oriented ideas encapsulation
  \item no side effects
  \item Estimation and Simulation  Policy Analysis Forecasting
  \end{itemize}
\end{itemize}

}
\end{frame}
\begin{frame}
\frametitle{Share code by design}
\begin{itemize}
\item Software development and maintenance tools
\item GitHub
\item API's
\item Think of sharing from the start
\item Plug and play
\end{itemize}
\end{frame}

\begin{frame}
  \frametitle{Other Benefits}
  \begin{itemize}
  \item reproducible results
  \item design to evolve ( expect breakage )
  \item Promotes Testing
  \item Eliminates Error Prone Tasks
  \item Division of labor and synergy  github
  \item Forking can be a good thing
  \end{itemize}
\end{frame}




\section{XML for  Model Definition and Manipulation}

\begin{frame}
  \frametitle{Star Concept}

  \begin{itemize}
  \item parse from language into some ``Abstract Syntax Tree'' (AST)
  \item post process from AST to other useful forms
  \item multiple parsers from
  \item facilitate sharing code as well as ideas
  \end{itemize}

\end{frame}

  \begin{frame}
    \frametitle{Intermediate Model Representation}
    \begin{itemize}
    \item Map multiple languages into xml form
    \item xml mapped into other languages or output
    \item \LaTeX
\item  Translation between model languages
\item factor out model manipulation and application from description
\item I chose XML open standard many free tools comprehensive perhaps too much so  XSL schema Xalan XSLT
    \end{itemize}
  \end{frame}


\begin{frame}{XML }

  \begin{itemize}
  \item Model Definition Language
  \item Tool Application Language
  \end{itemize}

  
\end{frame}


\begin{frame}
  \frametitle{Anecdotes}
  \begin{itemize}
  \item Lex/Yacc/AIM  now  ANTLR
  \end{itemize}
\end{frame}


\section{Symbolic Algebra for Code Generation}

\begin{frame}
  \frametitle{ Code Generation}
  


\begin{itemize}
\item Standard widely compilable cross platform
\item Fortran
\item C
\item MATLAB
\item \LaTeX
\item $\ldots$
\item Evolve to more specific
  \begin{itemize}
  \item Parallel
    \begin{itemize}
  \item GPU
  \item MPI
  \item $\ldots$
    \end{itemize}
  \end{itemize}
\end{itemize}

\end{frame}




\begin{frame}
  \frametitle{GPU Code Generation}
  
\end{frame}


\section{Examples}


\begin{frame}
  \frametitle{gitHub Links for DSGECodeGen Software}
  

  \begin{itemize}
  \item Package mathSmolyak` \href{https://github.com/es335mathwiz/AccelerateAMA.git}{https://github.com/es335mathwiz/AccelerateAMA.git}
  \item Package Accelerate` \href{https://github.com/es335mathwiz/AccelerateAMA.git}{https://github.com/es335mathwiz/AccelerateAMA.git}
  \item Package dsgeCodeGen` \href{https://github.com/es335mathwiz/dsgeCodeGen.git}{https://github.com/es335mathwiz/dsgeCodeGen.git}
\end{itemize}

\end{frame}



\subsection{Anisotropic Grid Smolyak}


\begin{frame}
  \frametitle{Anisotropic Grid Smolyak}
{\small
\cite{Judd2013} describes a technique for improving the performance of
the Smolyak method.
\begin{itemize}
\item Original select small proportion tensor product of complete grid to 
approximate a multidimensional function.
\item sparseGridEvalPolysAtPts[numVars\_Integer,approxLevel\_Integer,
ptGenerator\_Function,polyGenerator\_Function] 
\item sparseGridEvalPolysAtPts[listOfApproxLevels\_?ArrayQ,
ptGenerator\_Function,polyGenerator\_Function] 
\item chebyshevPolyGenerator::usage="chebyshevPolyGenerator[numPts\_Integer] generates chebyshev polynomials of the first kind with xx as the variable"
\item chebyshevPtGenerator::usage="chebyshevPtGenerator"
\item Other generators
  \begin{itemize}
  \item Isotropic 
  \item Anisotropic
  \end{itemize}
\end{itemize}
}
\end{frame}

\begin{frame}
  \frametitle{Outputs}
  \begin{itemize}
  \item $\hat{f}(x;b)=\sum_{n=1}^M b_n \Psi_n(\chi)$
  \item $
    \begin{bmatrix}
      f(\chi_1)\\ \cdots \\       f(\chi_M)
    \end{bmatrix}=
    % \begin{bmatrix}
    %   \hat{f}(\chi_1)\\ \cdots \\       \hat{f}(\chi_M)
    % \end{bmatrix}=
      \begin{bmatrix}
        \Psi_1(\chi_1) &\cdots &        \Psi_M(\chi_1)\\
\vdots&\ddots&\cdots\\
        \Psi_1(\chi_1)& \cdots   &      \Psi_M(\chi_1)
      \end{bmatrix}
      \begin{bmatrix}
        b_1\\ \vdots \\ b_m
      \end{bmatrix}= \mathcal{B} b
$
  \item The tuples of Smolyak grid points $\{\chi_1, \ldots , \chi_M\}$
  \item The vector of polynomial basis functions $\{\Psi_1, \ldots , \Psi_M\}$ (alternatively as calls to chebyshev library)
  \item The polynomials evaluated at the Smolyak grid points $\mathcal{B}$
  \end{itemize}
\end{frame}
\begin{frame}
  \frametitle{CodeGen}
  \begin{itemize}
  \item generate code for evaluating function at nodesa
  \item for applying inverted the matrix to get weights 
  \item for solving the linear system to get weights
  \item evaluating function with computed weights
  \item do for GPU, multicore, single core
  \item Uses \mma  and  its Foermat and Optimize packages
  \end{itemize}
  Gener
\end{frame}



\begin{frame}
  \frametitle{Anisotropic Smolyak Code Generation}
  
  \begin{itemize}
  \item Grid Specification
  \item \mma
  \end{itemize}
  \begin{gather*}
    f(x;\beta;\gamma)\\
\pdrv{f}{\beta},\pdrv{f}{x}
  \end{gather*}

\end{frame}


% \begin{frame}
%   \frametitle{Log Linearization}
  
% \end{frame}

% \begin{frame}
%   \frametitle{Impulse Response Functions}
  
% \end{frame}




\subsection{Anderson Moore Algorithm }

  \subsubsection{Anderson Mooore Algorithm Problem Statement}


\begin{frame}
  \frametitle{Problem Statement and Algorithm}
  
{\small  

Consider linear models of the form\cite{anderson10}:


\begin{gather}
\sum_{i=-\tau}^\theta{H_i x_{t+i}}= \Psi%\nonumber
z_{t}, \,\, t = 0,\ldots,\infty\label{eq:canonical}\\ \intertext{with initial conditions, if any, given by constraints of the form}%\nonumber
x_i  =  x^{data}_i,  i =  - \tau, \ldots, -1\label{eq:init}\\ \intertext{where both $\tau$ and $\theta$ are non-negative, and $x_t$ is an L dimensional vector 
of endogenous variables with}%\nonumber
\lim_{ t \rightarrow\infty} \|x_t\|   < \infty\label{eq:limit} %\nonumber
\end{gather}
{ and $z_t$ is a $k$ dimensional vector of exogenous variables.}

}

\end{frame}




\begin{frame}
The solution methodology entails 
Manipulating the left hand side of equation~\ref{eq:canonical} to obtain
 a state space transition matrix, $A$, along with
a set of auxiliary initial conditions, $Z$ for the homogeneous solution.
\begin{gather}
  Z
  \begin{bmatrix}
    x_{-\tau}\\ \vdots \\ x_{\theta}
  \end{bmatrix}=0 \,\,\,\,\text{and}\,   \begin{bmatrix}
    x_{-\tau+1}\\ \vdots \\ x_{\theta}
  \end{bmatrix}
=A   \begin{bmatrix}
    x_{-\tau}\\ \vdots \\ x_{\theta-1}
  \end{bmatrix}
\end{gather}

\end{frame}
\begin{frame}


 Computing the eigenvalues and vectors spanning 
the left invariant space associated with
large eigenvalues. 
\begin{gather}
 V A =   \mathcal{M}  V 
\end{gather}
with the eigenvalues of $ \mathcal{M}$ all greater than one in absolute value.

\end{frame}
\begin{frame}


 Assembling asymptotic
constraints, $Q$,   by combining the:
  \begin{enumerate}
\item  auxiliary initial conditions identified in the computation of the transition matrix and 
\item the invariant space vectors
  \end{enumerate}
\begin{gather}
  Q= 
  \begin{bmatrix}
    Z\\V
  \end{bmatrix}
\end{gather}

\end{frame}
\begin{frame}
 Investigating the rank of the linear space spanned by these asymptotic
constraints and,  when a unique solution exists, 
\begin{enumerate}
\item Computing the auto-regressive 
representation, $B$. 
\item Computing matrices, $\phi, F, \vartheta$ 
for characterizing the impact of the inhomogeneous
right hand side term.

\end{enumerate}
\end{frame}



% \begin{frame}
%   \frametitle{Anderson Moore Algorithm Components}
%   \begin{itemize}
%   \item Package AccelerateAMA` \href{https://github.com/es335mathwiz/AccelerateAMA.git}{on gitHub}

% \cite{anderson10}

%   \end{itemize}
% \end{frame}

\begin{frame}
  
  \subsubsection{A Simple Example}
Consider the model given by
  \begin{gather*}
    (1+r)v_t=v_{t+1}+ d_{t+1}\\
d_t=(1-\delta)d_{t-1}, \intertext{with} 0<r,\delta<1\\ \intertext{To apply the algorithms define}
x=
\begin{bmatrix}
  d_t\\v_t
\end{bmatrix}\,\,\,
H=
\begin{bmatrix}
  0&0&(1+r)&0&-1&-1\\ 0&-(1-\delta)&0&1&0&0
\end{bmatrix} 
\end{gather*}

\end{frame}

\begin{frame}
  
  \subsubsection{Transition Matrix and Auxiliary Initial
    Conditions}


\begin{gather*}
\intertext{The symbolic algebra implementation  computes the symbolic matrices}
Z=\begin{bmatrix}
 0&-(1-\delta)&0&1  
\end{bmatrix},
A=\begin{bmatrix}
0& 0& 1& 0\\0& 0& 0& 1\\ 0& 0& 1 + r&  \delta-1\\ 0& 0& 0& 1 - \delta
\end{bmatrix}
  \end{gather*}


\end{frame}



    \begin{frame}
  \frametitle{Eigenvalues and Eigenvectors}
 Two Simplifications facilitate the calculation


 Reduction in the dimension of A before computing eigenvalues
      \begin{gather*}
        A=
        \begin{bmatrix}
0& 0& 1& 0\\0& 0& 0& 1\\ 0& 0& 1 + r&  \delta-1\\ 0& 0& 0& 1 - \delta
        \end{bmatrix}
\rightarrow  a =
\begin{bmatrix}
1 + r& \delta-1\\0& 1 - \delta
\end{bmatrix} \intertext{The symbolic algebra implementation  computes the two symbolic eigenvalues}
\lambda_S=1-\delta,\lambda_L=1+r
      \end{gather*}
      \end{frame}

      \begin{frame}
        
 and computing only the left eigenvector associated with the single large root
      \begin{gather*} \intertext{The symbolic algebra implementation  computes the left eigenvector, e}
    NullSpace(    a^T-\lambda_L I) \rightarrow v=
      \begin{bmatrix}
                   \frac{\delta +r}{\delta -1} & 1        
      \end{bmatrix}
\intertext{ from which it is easy to construct the left eigenvector for $A$}
V=      \begin{bmatrix}
                  0&0& \frac{\delta +r}{\delta -1} & 1        
      \end{bmatrix}
      \end{gather*}
    \end{frame}



  \subsubsection{Assembling the Constraints and Computing the
    Autoregressive Representation}
 \begin{frame}
The symbolic algebra implementation  combines the constraints
    \begin{gather*}
      Q=
      \begin{bmatrix}
        Z\\V
      \end{bmatrix}=
      \begin{bmatrix}
         0&-(1-\delta)&0&1  \\
0&0&                   \frac{\delta +r}{\delta -1} & 1        
      \end{bmatrix}
    \end{gather*}
 The symbolic algebra implementation  computes the autoregressive representation
    \begin{gather*}
Q=
\begin{bmatrix}
  Q_L&Q_R
\end{bmatrix}=
\begin{bmatrix}
         0&-(1-\delta)&0&1  \\
0&0&                   \frac{\delta +r}{\delta -1} & 1        
\end{bmatrix} \intertext{leads to }
      B=
      \begin{bmatrix}
0& \frac{(\delta-1 )^2}{\delta + r}\\0& 1- \delta
      \end{bmatrix},
\phi=
\left(
                 \begin{array}{cc}
                  \frac{1}{\text{r}+1} & \frac{1-\delta }{\delta +\text{r}} \\
                  0 & 1
                 \end{array}
                 \right),
F=\left(
                 \begin{array}{cc}
                  \frac{1}{\text{r}+1} & \frac{1}{\text{r}+1} \\
                  0 & 0
                 \end{array}
                 \right)
    \end{gather*}
  \end{frame}

  \begin{frame}
    \frametitle{Practicality}
    To assess the practicality of obtaining symbolic solutions, I developed
    \begin{itemize}
    \item a Mathematica/Java program that
    \item reads  Dynare 4.2 models, gleans the equations,  parameters, 
variables and   initial values for parameters and
\item 
applies the symbolic algebra algoritims to compute analytic 
autoregressive solutions for the model.
\item For nonlinear models, the Mathematica  code attempts to symbolically computes 
steady states. The code
\item linearizes non-linear models using either the symbolic solution or a 
symbolic representation of a non-analytic steady state value.
\end{itemize}
\end{frame}
\begin{frame}
\begin{itemize}
\item I apply this program to a collection of representative models taken from 
\href{http://www.dynare.org/documentation-and-support/faq/basics}{the Dynare distribution,}
\href{http://www.nviegi.net/research/dsge.htm}{``Modelling and Simulating DSGE Models with Dynare in Octave'' by Nicola Viegis} and 
\href{http://homepages.nyu.edu/~ts43/research/AP_tom16.pdf}{``Practicing Dynare''\cite{bhandari10}}
    \end{itemize}
  \end{frame}

  \begin{frame}

    \subsubsection{The Models }
As shown in table \ref{tab:noEst} the models vary in size and in the number of parameters used in their
definition. Some of the models are nonlinear.

%\input{../code/modDims.tex}

  These nonlinear solutions have place holders for the steady state value.
The Mathematica code linearizes nonlinear models about a symbolic steady state.
Table \ref{tab:preEvalTimes} presents computation times for the steps
preceeding the eigenvalue calculations.
The symbolic algebra programs can deliver analytic solutions for
each of the transition matrices. This includes 17 equation ifsi and and 14 equation fs2000ns which required
about 3 minutes and a one half minute respectively. All the other models required less than a second.



%\input{../code/preEvalTimes.tex}



Symbolic algebra techniques deliver complete 
analytic solutions for each of the linear
models except for the BGGViegi model. 
I set a maximum of ???one hour??? for each stage of the computation.
For the BGGViegi model, the Eigenvector computation 
step took more than an hour before aborting this and each of the subsequent 
steps.  
The machine is a 2.93GHz  24 core
Intel(R) Xeon(R) CPU.\footnote{(peaklx13)}

%\input{../code/ssSolnTimes.tex}

%\input{../code/allCompTimes.tex}

Table \ref{tab:linCompTimes} presents the time required to compute analytic 
solutions. 
The second column presents the time to compute the state space 
transistion matrix, A. Column three presents the time required to determine
the eigenvalues and eigenvectors that help characterize the asymptotic
constraint matrix, Q. The fifth column presents the time required to
compute the B matrix characterizing the linear rational expectation  model
solution.  The sixth column characterizes the time required to compute
the observable structure matrix, S.  

FindRoot uses random numbers for variables not given by symbolic expression.
On rare occasion these do not converge except for ifs1 and fs2000
 which often has convergence problems. 
FindRoot solution times were always trivially small.

%\input{../code/linearCompTimes.tex}

Table \ref{tab:linMexTime} reports the number of seconds required for 1000 evaluations of
the matlab implementation of the Anderson-Moore algorithm (SPAmalg).
I have written a Mathematica program that creates a matlab executable file
a MEX file for computing the
Analytic mex reports the time required for computing the B matrix 1000 times.
The fourth column presents  the second column  divided by column three.
The fifth colum reports 
the number of function evaluations required before the speed of the
symbolic computing makes up for the overhead of
computing the symbolic solution.
\begin{gather*}
  N=\frac{P}{t_N-t_S}
\end{gather*}
where $t_N$ is the time for the numeric computation of the $B$ matrix,
$t_S$ is the time for the symbolic computation of the $B$ matrix, and
$P$ is the time needed to compute the symbolic B matrix.
]
The table shows that for small linear models, 
the analytic mex file is generally more than 100 times
faster than the fastest matlab implementation for solving linear rational
expectations models and that the time for computing the analytic is
made up for by this speed in under a dozen function evaluations.


%\input{../code/linCompMLB.tex}


Table \ref{tab:nonLinTimes} shows times to compute solutions for non linear
models. Non linear model computations require computation of the steady state.
I recomputed steady states for both the analytic and the matlab linear 
rational expectations solution.



%\input{../code/nonLinearCompTimes.tex}

Table \ref{tab:nonLinMexTime} shows that
speedups are comparable to speedups for linear models. But since the 
initial computation overhead for computing the solutions is larger, the
breakeven number of function evaluations is also larger for non liner models.

    
With nonlinear models comes the added complication of linearizing these models.
This can also typically be done using symbolic algebra.

One has the choice of simplifying the symbolic algebra code. This can be
time consuming or impossible.  The benefit may outweigh the cost.
One can compare the susequent execution times with and without Simplification.


%\input{../code/nonLinCompMLB.tex}

    
  \end{frame}

\begin{frame}
  \frametitle{Matlab Code Comparisons}
  
\end{frame}




\begin{frame}
  \frametitle{sympy Applications}
  
  \begin{itemize}
  \item Summer Intern Implementing AMA in python using sympy
  \item All necessary components there
    \begin{itemize}
    \item QRDecomposition
    \item LUDecomposition
    \item Eigenvalue and Eigenvector  NullSpacen
    \end{itemize}
  \item Mathematica more comprehensive Matrix manipulation
  \item Expect same bottle neck
  \end{itemize}
\end{frame}



\section{Future Directions}

\begin{frame}
  \frametitle{Future Directions}
  \begin{itemize}
  \item Talking out of my experiences
  \item Computers will someday write better code than we do
  \item Job to accurately describe what we want  (declarative programming)
  \item Symbolic, Functional and Logic programming may be useful Procedural
a dead end
\item Where are we computing
\item playing nicely with fellow coders will be good practice for smart 
machines that code for us
  \item at present my website is not a good example of good practice
    \begin{itemize}
    \item Testing
    \item Well defined API's
    \item Examples
    \item Predictable standardized Predictable Navigable structure
    \end{itemize}
  \end{itemize}
\end{frame}




\begin{frame}
  \frametitle{Bibliography}
  \bibliographystyle{authordate4}
\bibliography{anderson,files}


\end{frame}


\begin{frame}{MIT Hints}
  \href{http://web.mit.edu/rsi/www/pdfs/beamer-tutorial.pdf}{mit beamer slide how to}
\end{frame}
\begin{frame}

  \frametitle{Things to do}
  \begin{itemize}
\item Dynare incorporation
\item XML applications
  \begin{itemize}
  \item description
  \item application
  \end{itemize}
\item generate C and fortran
\item multicore and GPU ( use smolyak for GPU ) Format smarter than Chebyshev libraries?  libraries GPU
\item log linearization 
\href{http://web.mit.edu/14.452/www/pdf/rbc.pdf}{RBC notes}
\item judd maliar
\item discontinuous smolyak
\item anderson moore
\item occassionally binding constraints
\item continuous time
\item other good Mma over the years? check github
\item perturbation  rmat then balance or multicore
\item power series for impulse response function
\item symbollic function evaluated at grid points?
\item GSM code /msu/scratch/m1gsa00/learnProjection/proto/authoritative/newGsmCode.mth
\item<1>{this is overlay}\only<2->{aha}
\item<2>{what do you know}
\item<3>{what do you 3}
\item<5>{what do you 5}
\end{itemize}
 \begin{itemize}
      \item\only<1>{World peace}\only<2->{\sout{World peace}}
      \item<2-> Peace in our neighbourhood
    \end{itemize}
\end{frame}



\end{document}
